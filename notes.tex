%%fakesection
\documentclass{article}
\usepackage[usenames,dvipsnames]{color}
\usepackage{soul}
\usepackage{fullpage}
\usepackage{xcolor}
\usepackage{hyperref}
\usepackage{pifont}
\usepackage{gnuplottex}
\usepackage{float}


\let\Item\item
\renewcommand\item{\normalcolor\Item}

\newcommand\ml{\color[RGB]{153, 150, 204}} %may be later
\newcommand\later{\color[RGB]{153, 204, 150}} %later
\newcommand\nn{\color[RGB]{124, 124, 255}} %no need to do
\newcommand\done{\color[RGB]{129, 180, 185} \ding{52} }
\newcommand\now{\color[RGB]{255, 0, 0}} %current

\begin{document}

\section{doing}
\subsection{output energy conservation}
\begin{itemize}
  \item \done tested creating directory with boost
  \item \done create a directory for each rank
  \item \done create a directory for each trajectory
  \item \done store trajectory
  \item create a switch flag in the config.lua file.
    \begin{itemize}
      \item pass parameters through ...
    \end{itemize}
\end{itemize}
\subsection{spin-boson case}
\begin{itemize}
  \item \done hamiltonian
  \item \done d\_hamiltonian
  \item check energy conservation
\end{itemize}
\section{to do}
\subsection{templated hamiltonian}
form: $v1*n1+v2*n2+v12*n1n2+vb$
\section{to do priority unsorted}
\subsection{test cases}
\begin{itemize}
  \item same energy
  \item same position
  \item no bath
  \item no coordinate
\end{itemize}
\subsection{move make trajs, integrate, avg and el to one function}
\section{plan}
%%%%%%%%%%%%%%%%%%
\section{done}
\subsection{mpi}
Sat Oct 26 17:48:08 PDT 2013
\subsection{output time in ave\_n}
Sat Oct 26 17:46:27 PDT 2013
\begin{itemize}
  \item \done add devision by mpi\_rank
  \item \done take time from the second variable in accumulate's function
\end{itemize}
\subsection{average at each step}
Sat Oct  5 16:48:52 PDT 2013
\begin{itemize}
  \item the function takes a state and outputses state(t+1) and average observables (p,q), Pnn
  \item may be make it a class that keeps a state or a functor
  \item output new state only if needed if not needed just keep it as tmp
\end{itemize}
algorithm:
\begin{enumerate}
  \item \done create a functor that propagates the state by dt. has parameters as the functor's state
    \begin{enumerate}
      \item \done vector of times
      \item \done transform time into vvd of observables
      \item \done transform each time step into a zip\_iterator of ave\_pq and ave\_nn
      \item \done for each do step, run a trajectory
      \item \done transform each step to run the same trajs
      \item \done ave states
      \item \done ave nn
      \item \done change a trajectory into running mutiple trajectories
    \end{enumerate}
  \item \done average p and q
  \item \done average n1, n2, pnn
\end{enumerate}
what is the structure of do step? should i use for-loop or stl?
\begin{itemize}
  \item for-lop would execute each step
  \item create a vector of times then for each time, 
  \item transform time vector to ave obervables
  \item do a step 
  \item use zip iterator to simultaniously get ave\_traj and ave\_nn
\end{itemize}
\subsection{reduce compilation time by refractoring main}
Fri Sep 27 13:43:32 PDT 2013
\begin{itemize}
  \item \done load config file
  \item \done create initial states
\end{itemize}
\end{document}
